\chapter{反思与总结}

总结本阶段实验,第一项实验给我留下了最深刻的印象。在我之前的认识中,图像的预处理在不同应用领域应当都有最常用的实现方式,甚至于可证明的最优解(State of art)。然后发现这里面的水还是太深了……要考虑的内容比神经网络相关的具体太多,而不是虚无缥缈地调参。同时,我甚至一直以为色彩转灰度只是简单的平均数计算,但对比的图像及其相应的结果都让人大吃一惊。这说明了人眼很容易被图像诓骗,但它又经常对变化敏感。那么,文献收下了,咱慢慢看。

在整理实验报告的时候才注意到滤波器还有很重要的一点没有考虑到,即其作用范围(滤波子的大小)。中值滤波选用稍大的滤波子时,可能会对该数据集的噪声有显著提升的抑制作用。但噪声最好也得用个工具,从信号的角度出发,量化测量一下,不然又得用肉眼看?这可不太好。

在本实验中,数据集有过拟合倾向,因此还应考虑选用其他人脸数据集进行测试,这样或许也能进一步验证我的猜想。不过本次实验也让我迷迷糊糊地认识了协方差矩阵这一个重要概念,老朋友以后见!

感谢老师、助教、同学和开源资料,以及所引参考文献的各作者。该实验的完成缺少不了你们的贡献。