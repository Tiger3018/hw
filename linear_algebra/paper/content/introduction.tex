
\chapter{绪论}

\section{机器学习的数学工具}
近年大火的机器学习和人工智能技术,均基于计算机的高计算性能实现,其内容主要包含对数据的处理、分类和预测,以及其对应的算法实现。

具体来说,机器学习\textbf{算法实现}的理论基础多为线性代数和概率论、统计论的相关知识,如回归分析、决策树、极值分析和最值分析、集成和聚类学习等。这些算法再经由计算机的软硬件算法,对处理过程加以优化,最终形成可调用的库供用户使用。例如,计算机的图形计算单元(GPU),其单指令多线程(以万为单位)的特性,决定了其适用于多维数据的快速计,并部署上述的各类算法。经优化,现在的GPU算力通常以GFLOPS(Giga-FLOPS, floating-point operations per second, 每秒10亿次的浮点运算数),实现了相较于CPU更快的发展,同时也促进了机器学习的革新进展。

所谓的\textbf{模型}一般用于解决特定性的问题,如YOLO用于机器视觉的多目标识别及分类,及BERT用于自然语言的语句识别及分类。当然,部分算法的代码实现,由于结合了多种算法思想,故也会被称作基本模型,例如概率图模型。前述模型则整合了多种基本模型,以应对复杂的现实问题。因此,非通用类的人工智能均由机器学习实现,本质仍是对数据的处理。\cite{周志华_2016, Sulmont_2019}

与之相对应,通用类的人工智能有机器学习(即数据处理)和生物神经模拟两类研究方向,但目前仍无突出优势和应用。

\section{人脸识别的社会应用}

人脸识别领域是机器学习视觉处理的重要组成部分,也是应用最广泛、社会争议最大的。虽然其名字中只有“识别”两个字,但其内容却包含了识别、分类甚至于特征替换等。经典的特征替换应用实例是DeepFake,这样一个模型可以将视频内的人脸进行替换,并在相当程度上让人类信以为真。虽然机器学习又可以通过该模型的训练规律,反向判断应用该模型“换脸”的视频,但只需稍微应用对抗学习的方法,机器便能自身演化形成独特的替换算法,而不被检测出规律性。\cite{Guera_2018}像这种引起社会恐慌的例子还有很多,在此就不一一列举。

不可否认的是,各类人脸识别系统在近几年如雨后春笋发展起来,但在真实场景的使用中,受限于硬件、成本等条件,这些系统使用的多半是传统的整体分类算法,其发展更多依靠的是数字处理能力的上升,以及智能终端的广泛普及,并已带来相当多的法理伦理问题。本次实验中,程序对数据集进行特征值提取处理后,再应用分类算法进行人脸识别的算法,也是一种简单的整体分类算法。以openCV库中人脸识别的类 \verb|FaceRecognizer| 为例,其包含Eigenface、Fisherface整体(代数特征)算法,以及LBPH局部特征提取算法。这些算法的不足和优化方向均是与脸部特征无关的图像特征,如人脸朝向、光照方向等。

商用级的人脸识别和应用还是以整体(Holistic)识别算法为主,并使用主成分分析(PCA)、独立成分分析(ICA)、线性判别分析(LDA)等作为识别不相关量,并进行降维的算法。在前沿研究和大数据处理方向,基于神经网络的人脸识别则经过卷积神经网络(CNN)等网络模型,被训练为提取偏好特征的特性(Feature)识别算法。基于神经网络的方案又被称为深度学习,它比传统机器学习需要更丰富多元的数据量,同时需要更多的计算性能,因此也更受人们的关注和争议。\cite{Zhao_2003, Anand_2017}

本次实验中要求以可交互界面(可视化用户界面,GUI)实现的特征脸(Eigenface)分类算法,使用的核心算法是主成分分析。实验成果参考了对应的算法,根据\autoref{cha:fr}详述的系统框架实现了相应的目标。具体的算法原理及性能、参数比较,将在\autoref{cha:bg}讨论。此外,为了对图像处理有基础的了解,本次实验还要求对图片进行色彩空间和空域滤波的预处理,具体结果将在\autoref{cha:re}讨论。








