\chapter{附\hskip\ccwd{}录}
\section{程序源代码}

\begin{figure}[htb]
	\dirtree{%
		.1 \myfolder{pink}{工作文件夹}.
		.2 \myfolder{cyan}{class1.ipynb}.
		.2 \myfolder{cyan}{recognition.ipynb}.
		.2 \myfolder{cyan}{hw1.py}.
		.2 \myfolder{cyan}{hw2.py}.
		.2 \myfolder{cyan}{dataset}.
		.3 \myfolder{lime}{s1}.
		.3 \myfolder{lime}{...}.
		.3 \myfolder{lime}{test}.
	}%\dirtree
\caption[]{软件源代码文件结构图示。}
\label{fig:filetree}
\end{figure}

另可见\url{https://github.com/Tiger3018/hw/tree/HEAD/linear_algebra}。

\verb|hw1.py|: 图像的灰度处理
\inputminted{python}{./code/hw1.py}
\verb|hw2.py|: 图像的滤波处理
\inputminted{python}{./code/hw2.py}
\verb|class1.ipynb|: 图像的主成分分析,训练模块。
\inputminted{python}{./code/class1.ipynb}
\verb|recognition.ipynb|: 分类模块:可视化的训练数据读取及识别分类
\inputminted{python}{./code/recognition.ipynb}

\section{在线运行}

可以访问\url{https://mybinder.org/v2/gh/tiger3018/hw/89041d1?labpath=linear_algebra\%2class1.ipynb},快速实现视频中展示的所有软件功能。只需注意:

\begin{itemize}
    \item 需要一个稳定的网络环境。
    \item 分配得到的虚拟环境内存需要为2GB。当分配到的内存仅有1GB时,不能正常生成训练结果。
    \item 请将训练模块中的计时部分(\verb|%timeit|)注释(\verb|#|)。出现的计时部分在云端运行一次,分别需要约30秒和20秒。
\end{itemize}

要生成训练结果\verb|train.npz|,请完全运行一次\verb|class1.ipynb|。在生成训练结果或上传训练结果后,再完全运行一次\verb|recognition.ipynb|,即可展示可交互界面。

%\begin{table}[htb]

%\begin{equation}
%\alpha\beta\gamma\delta\epsilon\varepsilon\zeta\eta = CD\Gamma\varGamma Z
%\end{equation}\eqlist{附录中的公式编号2,双语}[Equation name in English B]

%测试用途:theequation值为:\theequation ,thefigure值为:\thefigure ,thetable值为:\thetable
